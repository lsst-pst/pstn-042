
\begin{abstract}
 
The Vera C. Rubin Observatory will conduct a 10-year survey of the night sky visible from Chile to address many questions at the frontier of astrophysics and cosmology research.
The survey will collect many petabytes of image data, which will be processed to generate a diverse set of derived data products, each with stringent set of requirements flowed down from science drivers.
Meanwhile, the observatory itself is a collection of complex subsystems that must function together to deliver high-quality scientific images.
We describe here the systems engineering approach adopted by the Rubin Observatory construction project used to verify each step of the system integration, and to monitor the system performance over time.
We focus on the application of these tools to science verification and validation during commissioning.

\end{abstract}

